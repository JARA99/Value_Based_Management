% Title
\begin{center}
    {\huge \textbf{Listado de ecuaciones}}\\
    \vspace{0.5cm}
    {\large \textit{Value Based Management}}
\end{center}

% Begin multicolumn layout
% \begin{multicols}{2}
% \columnratio{0.35}
% \begin{paracol}{2}

\section{Lista de variables}
Para desarrollar el listado de ecuaciones se utilizarán las siguientes variables:

\begin{itemize}
    \item [$\eta$ \---] \textit{Net Operating Profit After Tax}
    \item [$t$ \---] \textit{Tax rate}
    \item [$c$ \---] Capital
    \item [$a$ \---] Activos
    \item [$p$ \---] Pasivos
    \item [$s$ \---] Ventas
    \item [$r$ \---] \textit{Return On Invested Capital}
    \item [$w$ \---] \textit{Weighted average cost of capital}
    \item [$e$ \---] \textit{Economic Value Added}
    \item [$\delta$ \---] \textit{Operating income}
    \item [$I$ \---] Inversión neta
    \item [$\gamma$ \---] \textit{Free cash flow}
    \item [$f_n$ \---] Factor de descuento a valor presente del año $n$.
    \item [$F$ \---] Factor de descuento a valor presente de una perpetuidad.
    \item [$g$ \---] Tasa de crecimiento
    \item [$V$ \---] Valor
\end{itemize}

\section{Valor presente}

Valor presente de un único pago recibido en $n$ años.
\begin{equation}
    f_n = \frac{1}{\left(1+w\right)^n}
\end{equation}

Valor presente del flujo a perpetuidad con un crecimiento $g$, recibido a partir del año $n$.

\begin{equation}
    F = \frac{1}{w-g}\cdot\frac{1}{\left(1+w\right)^{n-1}}
\end{equation}



\section{Valuacion por EVA}

\subsubsection*{NOPAT}

\begin{equation}
    t = \frac{\textrm{\textit{Earning before tax}}}{\textrm{\textit{Tax provision}}}
\end{equation}

\begin{equation}
    \eta = \delta(1-t)
\end{equation}

\subsubsection*{Capital}
\begin{equation}
    c = a_{total} - \left(a_{no.op.} + p_{gratuitos} \right)
\end{equation}

\subsubsection*{ROIC}
\begin{equation}
    r = \frac{\eta}{c}
\end{equation}

\subsubsection*{EVA}
\begin{equation}
    e = c(r-w)
\end{equation}

\subsubsection*{Valuación por EVA}
\begin{align*}
    V   &= \sum_{n=1}^\infty f_n \cdot e_n \\
        & + \left(a_{no.op.}+p_{gratuitos}\right) + c_1\\
    V   &= \sum_{n=1}^{m-1} f_n \cdot e_n + F \cdot e_m\\
        & + \left(a_{no.op.}+p_{gratuitos}\right) + c_1
\end{align*}


\section{Value drivers}

La descomposición en Value Drivers consiste en separar el ROIC en los elementos que componen los gastos operativos.

\begin{equation}
    \left(\frac{s}{s}+\frac{cost}{s}+\frac{op.ex.}{s}\right)\cdot \frac{s}{c}\cdot(1-t)
\end{equation}

\section{Valuación por flujos descontados}

\subsubsection*{Inversión neta}
\begin{equation}
    I_n = c_{n+1} - c_{n}
\end{equation}

\subsubsection*{Free cash flow}
\begin{equation}
    \gamma = \eta - I
\end{equation}

\subsubsection*{Valuación por free cash flow}

\begin{align*}
    V   &= \sum_{n=1}^\infty f_n \cdot \gamma_n \\
        & + \left(a_{no.op.}+p_{gratuitos}\right)\\
    V   &= \sum_{n=1}^{m-1} f_n \cdot \gamma_n + F \cdot \gamma_m\\
        & + \left(a_{no.op.}+p_{gratuitos}\right)
\end{align*}

% \end{multicols}
% \end{paracol}